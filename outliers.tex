\documentclass[aps,rmp, onecolumn]{revtex4}
%\documentclass[a4paper,10pt]{scrartcl}
%\documentclass[aps,rmp,twocolumn]{revtex4}

\usepackage[utf8]{inputenc}
\usepackage{amsmath,graphicx}
\usepackage{color}
%\usepackage{cite}

\newcommand{\bq}{\begin{equation}}
\newcommand{\eq}{\end{equation}}
\newcommand{\bn}{\begin{eqnarray}}
\newcommand{\en}{\end{eqnarray}}
\newcommand{\Richard}[1]{{\color{red}Richard: #1}}
\newcommand{\LH}{\mathcal{L}}

\begin{document}
\title{Detecting outliers in heterochronous phylogenetic trees}
\author{Richard Neher}
\date{\today}
\maketitle
Misdating of sequences, or sequences with many sequencing errors, commonly distort time scaled phylogenetic trees.
A common tactic to spot such sequences is to plot the root-to-tip distance as a function of time, reroot to optimize the correlation between them, and exclude tips that are far from the regression line.
This is what augur, treetime, and TempEst currently do.

The problem with this approach is that it is not very sensitive since it ignores phylogenetic relationships among the tips.
A sequence dated a year too early might still fall into the distribution of root-to-tip distances of that date, but is a clear outlier when compared directly to its neighbors in the tree.

To spot such outliers sensitively, we model the distribution in time for samples of a particular genotype $i$ as
\begin{equation}
    P(t|\tau_i, \sigma) = e^{-\frac{(t-\tau_i)^2}{2\sigma^2}}/\sqrt{2\pi\sigma^2}
\end{equation}
Here $\tau_i$ is the time when most samples of this genotype are around, $\sigma$ is the width of this distribution, which could correspond to the growth and decline of a variant or clade.
Different genotypes are phylogenetically related and the molecular clock constrains how the $\tau_i$ change along the tree.
For simplicity, we will models this as a Gaussian as well.
Referring to the parent of $i$ as $p_i$, we have for the full log-LH
\begin{equation}
    \LH  = \sum_i \left(\frac{(\mu(\tau_i - \tau_{p_i}) - d_i)^2}{2(d_i+1)} + \sum_{\alpha \in s_i} \frac{(t_\alpha-\tau_i)^2}{2\sigma^2} \right)
\end{equation}
where $d_i$ is the number of mutations between $i$ and $p_i$.
We want to optimize this with respect to the genotype timings $\tau_i$.
Diffentiating with respect to $\tau_k$, we have
\begin{equation}
    \partial_{\tau_k} \LH  = \frac{(\mu(\tau_k - \tau_{p_k}) - d_k)}{d_k+1} + \sum_{\alpha \in s_k} \frac{(\tau_k-t_\alpha)}{\sigma^2} -  \sum_{c\in k} \frac{(\mu(\tau_{c} - \tau_{k}) - d_c)}{d_c+1} = 0
\end{equation}
This is a sparse linear system that can be readily solved for $\tau_k$. This could also be solved analytically in a forward backward fashion.
It will be useful to define the average time $\bar{t}_i$ and the number of observations of genotype $i$ as $n_i$ to simplify the above to
\begin{equation}
    \partial_{\tau_k} \LH  = \mu\frac{(\mu(\tau_k - \tau_{p_k}) -  d_k)}{d_k+1} +  n_k\frac{(\tau_k-\bar{t}_k)}{\sigma^2} -  \mu\sum_{c\in k} \frac{(\mu(\tau_{c} - \tau_{k}) - d_c)}{d_c+1} = 0
\end{equation}

The resulting times can then be plugged into $\LH$ and we can optimize $\sigma$ and maybe $\mu$.
Once those are optimized, we can compare each nodes sampling time to its distribution.
If we did the forward-backward distributions, we could also look at the leave-on-out signal.

\section*{Forward-backward solution}
For a terminal node $k$, the optimal position given the position of the parent $\tau_{p_k}$ is
\begin{equation}
\tau_k =  \left(\frac{n_k}{\sigma^2} + \frac{\mu^2}{d_k+1}\right)^{-1}\left(\frac{n_k \bar{t}_k}{\sigma^2} + \mu\frac{\mu\tau_{p_k} + d_k}{d_k+1}\right) = a + b\tau_{p_k}.
\end{equation}
This expression weighs the evidence of the node being placed close to the average samples against the position of and the mutations relative to the parent.
A similar calculation can be done for internal nodes where we have additional contributions from the children where we plug in the expression for their optimal position given the parent.
\begin{equation}
    \mu\frac{(\mu(\tau_k - \tau_{p_k}) -  d_k)}{d_k+1} +  n_k\frac{(\tau_k-\bar{t}_k)}{\sigma^2} -  \mu\sum_{c\in k} \frac{(\mu(a_c + b_c \tau_{k} - \tau_{k}) - d_c)}{d_c+1} = 0
\end{equation}
Collecting all terms proportional to $\tau_k$, we find
\begin{equation}
\tau_k\left(\frac{n_k}{\sigma^2} + \frac{\mu^2}{d_k+1} + \mu^2\sum_{c\in k} \frac{1-b_c}{d_c+1}\right) =
\frac{n_k \bar{t}_k}{\sigma^2} + \mu\frac{\mu\tau_{p_k} + d_k}{d_k+1} + \mu \sum_{c\in k}\frac{\mu a_c - d_c}{d_c+1}
\end{equation}
Which can again be solved for $\tau_k$ and is a linear function of $\tau_{p_k}$.
\begin{equation}
    \tau_k =\left(\frac{n_k}{\sigma^2} + \frac{\mu^2}{d_k+1} + \mu^2\sum_{c\in k} \frac{1-b_c}{d_c+1}\right)^{-1}\left(
    \frac{n_k \bar{t}_k}{\sigma^2} + \mu\frac{\mu\tau_{p_k} + d_k}{d_k+1} + \mu \sum_{c\in k}\frac{\mu a_c - d_c}{d_c+1}\right) = a + b \tau_k
\end{equation}
At the root, the term from the parent is absent and their is no conditioning on $\tau_{p_k}$ anymore
\begin{equation}
    \tau_k =\left(\frac{n_k}{\sigma^2} + \mu^2\sum_{c\in k} \frac{1-b_c}{d_c+1}\right)^{-1}\left(
    \frac{n_k \bar{t}_k}{\sigma^2} + \mu \sum_{c\in k}\frac{\mu a_c - d_c}{d_c+1}\right) = a + b \tau_k
\end{equation}
Once all the root $\tau$ is known, all other $\tau_k$ can be calculated in one backward pass.










\end{document}